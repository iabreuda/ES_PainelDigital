\documentclass[a4paper, 12pt]{article}

\usepackage[portuges]{babel}
\usepackage[utf8]{inputenc}
\usepackage{amsmath}
\usepackage{indentfirst}
\usepackage{blindtext}
\usepackage{graphicx}
\usepackage[hidelinks]{hyperref}
\usepackage{gensymb}
\usepackage[top=2cm, bottom=2cm, left=3cm, right=2cm]{geometry}
\usepackage[table]{xcolor}
\usepackage{pbox}
\author{Time Peru}

\title{Painel Digital - Engenharia de Software 2016/02}
\begin{document}
	
\begin{minipage}[c][3cm][c]{3cm}
	\includegraphics[height=3cm]{img/ufrj_logo.PNG} 
\end{minipage}
\begin{minipage}[c][3cm][c]{10cm} 
	Universidade Federal do Rio de Janeiro \\
	EEL873 - Engenharia  de Software \\
	Projeto de Disciplina: Sistema Web
\end{minipage}

\begin{table}[ht]
	\rowcolors{1}{}{}
	\centering
	
	\begin{tabular}{p{6cm}p{10cm}}
		\hline
		\cellcolor{gray} & \cellcolor{gray}\\
		\hline		
		\pbox{20cm}{Nome do Projeto: \\ Painel Digital}
		&\pbox{20cm}{Data de Solicitação: \\ 15 de Setembro de 2016}\\[2ex]
		\hline
		\pbox{20cm}{Responsável: \\ Time Peru}\\[2ex]
		\hline
		\pbox{20cm}{Solicitante: \\ Guilherme Horta Tavassos}
		&\pbox{10cm}{Clientes: \\ Fábio Farzat, Guilherme Travassos, Hilmer Neri, Talita Ribeiro}\\[4ex]
		\hline
		\cellcolor{gray} & \cellcolor{gray}\\
		\hline
	\end{tabular}
\end{table}%

\begin{table}[ht]
	\rowcolors{1}{}{}
	\centering
	
	\begin{tabular}{p{2cm}p{9cm}p{2cm}p{2cm}}
		\hline
		\cellcolor{gray}Data&\cellcolor{gray}Comentário&\cellcolor{gray}Autor&\cellcolor{gray}Versão  \\
		\hline
		2016/09/28&Levantamento inicial de requisitos &Igor&0.1\\
		\hline

	\end{tabular}
\end{table}%
\vspace{100pt}
\begin{center}
	\huge{Painel Digital - Time Peru}
	\vspace{95pt}
\end{center}
\newpage
\tableofcontents
\newpage
\section{Visão}
O problema da comunicação interna aos docentes, discentes, funcionários e visitantes da UFRJ sobre os eventos e acontecimentos atuais nos campi por onde transitam afetam docentes, discentes, funcionários e visitantes da
universidade cujo impacto é a comunicação pouco efetiva e eficiente das atividades, cursos e eventos programados para
acontecer nos diferentes centros da universidade durante determinados períodos do ano. Uma boa solução seria um painel digital interativo no qual possam ser disponibilizados informações sobre eventos, cursos e demais atividades que acontecem na UFRJ em determinados períodos do ano.

\subsection{Escopo do Projeto}
O projeto consiste em desenvolver um sistema para gerenciamento de
anúncios de atividades, cursos, eventos e vagas para bolsas e estágios relacionados
com a UFRJ. Esses anúncios deverão ser cadastrados por interessados em divulgação
de eventos, cursos e demais atividades que necessariamente ocorrerão dentro dos
diferentes campi da universidade. Informações sobre os anúncios (detalhes, data e
local) bem como informações de período em que esses anúncios ficarão disponíveis
para visualização e locais (centros) em que esses informativos possam ser de maior
interesse do público deverão ser informados. É previsto que administradores possam
publicar e despublicar anúncios a qualquer momento, mesmo que esses ainda não
tenham expirado sua data de publicação. A visualização desses anúncios ocorrerá via
painéis digitais a serem instalados em diversos pontos da UFRJ. A atualização dos
anúncios ocorrerá de forma automática baseando-se em configuração temporal
periódica indicada por administradores do sistema e nas informações cadastradas
para cada anúncio referentes a período e locais para publicação.

O painel digital deverá ser interativo, o que significa dizer que o usuário
poderá selecionar tipos de anúncios de seu interesse para visualização e ainda
interagir com o anúncio indicando sua reação (curtir, amar, enfurecer…) em relação a
ele.
\subsection{Escopo não incluído no Projeto}
Toda a gerencia dos dados referentes as bolsas serão importados diretamente do sistema de gerenciamento e ofertas de bolsas e estágios da UFRJ. A inclusão/exclusão/modificação deverá ser feita no sistema integrado, o Painel digital será responsável somente pela visualização destas informações.
\newpage
\subsection{Envolvidos no Projeto}
\begin{table}[ht]
	\rowcolors{1}{}{}
	\centering
	
	\begin{tabular}{p{7cm}p{7cm}}
		\hline
		\cellcolor{gray}Nome&\cellcolor{gray}Papel  \\
		\hline
		Anderson de Souza&Engenheiro de Software Peru\\
		Carlos Felipe&Engenheiro de Software Peru\\
		Igor Abreu&Engenheiro de Software Peru\\
		Luis Octavio&Engenheiro de Software Peru\\
		Renan Fasolato&Engenheiro de Software Peru\\
		Fábio Farzat&Cliente\\
		Guilherme Travassos&Cliente\\
		Hilmer Neri&Cliente\\
		Talita Ribeiro&Cliente\\
		Ana Paula&Time de Integração Argentina\\
		Carlos Eduardo&Time de Integração Argentina\\
		Hellen Pereira&Time de Integração Argentina\\
		Luis Felipe&Time de Integração Argentina\\
		Rafael Gonçalves&Time de Integração Argentina\\		
		\hline
	\end{tabular}
\end{table}
\subsection{Glossário}
\begin{table}[ht]
	\rowcolors{1}{}{}
	\centering
	
	\begin{tabular}{p{2cm}p{12cm}}
		\hline
		\cellcolor{gray}Termo&\cellcolor{gray}Descrição  \\
		\hline
		RFXX&Requisitos funcionais, seguido por um índice numérico sequencial\\
		RNFXX&Requisitos não funcionais, seguido por um índice numérico sequencial\\	
		\hline
	\end{tabular}
\end{table}
\section{Requisitos do Sistema}
\subsection{Requisitos Funcionais}
\begin{table}[ht]
	\rowcolors{1}{}{}
	\centering
	
	\begin{tabular}{p{2cm}p{7cm}p{2cm}p{2cm}}
		\hline
		\cellcolor{gray}Código&\cellcolor{gray}Descrição&\cellcolor{gray}Status&\cellcolor{gray}Prioridade  \\
		\hline
		-&-&-&-\\
		\hline
	\end{tabular}
\end{table}%
\newpage
\subsection{Requisitos Não Funcionais}
\subsubsection{Requisitos de Comunicação de Dados, Interface e Interoperabilidade}
	\begin{table}[ht]
		\rowcolors{1}{}{}
		\centering
		
		\begin{tabular}{p{2cm}p{7cm}p{2cm}p{2cm}}
			\hline
			\cellcolor{gray}Código&\cellcolor{gray}Descrição&\cellcolor{gray}Status&\cellcolor{gray}Prioridade  \\
			\hline
			-&-&-&-\\
			\hline
		\end{tabular}
	\end{table}%
\subsubsection{Requisitos de Confiabilidade}
\begin{table}[ht]
	\rowcolors{1}{}{}
	\centering
	
	\begin{tabular}{p{2cm}p{7cm}p{2cm}p{2cm}}
		\hline
		\cellcolor{gray}Código&\cellcolor{gray}Descrição&\cellcolor{gray}Status&\cellcolor{gray}Prioridade  \\
		\hline
		-&-&-&-\\
		\hline
	\end{tabular}
\end{table}%
\subsubsection{Requisitos de Desempenho e Robustez}
\begin{table}[ht]
	\rowcolors{1}{}{}
	\centering
	
	\begin{tabular}{p{2cm}p{7cm}p{2cm}p{2cm}}
		\hline
		\cellcolor{gray}Código&\cellcolor{gray}Descrição&\cellcolor{gray}Status&\cellcolor{gray}Prioridade  \\
		\hline
		-&-&-&-\\
		\hline
	\end{tabular}
\end{table}%
\newpage
\subsubsection{Requisitos de Disponibilidade}
\begin{table}[ht]
	\rowcolors{1}{}{}
	\centering
	
	\begin{tabular}{p{2cm}p{7cm}p{2cm}p{2cm}}
		\hline
		\cellcolor{gray}Código&\cellcolor{gray}Descrição&\cellcolor{gray}Status&\cellcolor{gray}Prioridade  \\
		\hline
		-&-&-&-\\
		\hline
	\end{tabular}
\end{table}%
\subsubsection{Requisitos de Manutenibilidade}
\begin{table}[ht]
	\rowcolors{1}{}{}
	\centering
	
	\begin{tabular}{p{2cm}p{7cm}p{2cm}p{2cm}}
		\hline
		\cellcolor{gray}Código&\cellcolor{gray}Descrição&\cellcolor{gray}Status&\cellcolor{gray}Prioridade  \\
		\hline
		-&-&-&-\\
		\hline
	\end{tabular}
\end{table}%	
\subsubsection{Requisitos de Portabilidade}
\begin{table}[ht]
	\rowcolors{1}{}{}
	\centering
	
	\begin{tabular}{p{2cm}p{7cm}p{2cm}p{2cm}}
		\hline
		\cellcolor{gray}Código&\cellcolor{gray}Descrição&\cellcolor{gray}Status&\cellcolor{gray}Prioridade  \\
		\hline
		-&-&-&-\\
		\hline
	\end{tabular}
\end{table}%
\newpage
\subsubsection{Requisitos de Segurança}
\begin{table}[ht]
	\rowcolors{1}{}{}
	\centering
	
	\begin{tabular}{p{2cm}p{7cm}p{2cm}p{2cm}}
		\hline
		\cellcolor{gray}Código&\cellcolor{gray}Descrição&\cellcolor{gray}Status&\cellcolor{gray}Prioridade  \\
		\hline
		-&-&-&-\\
		\hline
	\end{tabular}
\end{table}%
\subsubsection{Requisitos de Usabilidade}
\begin{table}[ht]
	\rowcolors{1}{}{}
	\centering
	
	\begin{tabular}{p{2cm}p{7cm}p{2cm}p{2cm}}
		\hline
		\cellcolor{gray}Código&\cellcolor{gray}Descrição&\cellcolor{gray}Status&\cellcolor{gray}Prioridade  \\
		\hline
		-&-&-&-\\
		\hline
	\end{tabular}
\end{table}%
\subsubsection{Restrições de Projeto e Tecnológicas}
\begin{table}[ht]
	\rowcolors{1}{}{}
	\centering
	
	\begin{tabular}{p{2cm}p{7cm}p{2cm}p{2cm}}
		\hline
		\cellcolor{gray}Código&\cellcolor{gray}Descrição&\cellcolor{gray}Status&\cellcolor{gray}Prioridade  \\
		\hline
		-&-&-&-\\
		\hline
	\end{tabular}
\end{table}%
\newpage
\subsubsection{Restrições Legais}
\begin{table}[ht]
	\rowcolors{1}{}{}
	\centering
	
	\begin{tabular}{p{2cm}p{7cm}p{2cm}p{2cm}}
		\hline
		\cellcolor{gray}Código&\cellcolor{gray}Descrição&\cellcolor{gray}Status&\cellcolor{gray}Prioridade  \\
		\hline
		-&-&-&-\\
		\hline
	\end{tabular}
\end{table}%
\section{Referências}
\begin{table}[ht]
	\rowcolors{1}{}{}
	\centering
	
	\begin{tabular}{p{4cm}p{2cm}p{7.5cm}}
		\hline
		\cellcolor{gray}Título do Documento&\cellcolor{gray}Versão&\cellcolor{gray}Localização  \\
		\hline
		-&-&-\\	
		\hline
	\end{tabular}
\end{table}
\section{Concordância do Cliente}
Concordo com os requisitos listados neste documento. Estou ciente de que o planejamento do projeto será realizado com base nesses requisitos aprovados.
\\\\
\textbf{Nome:}
\\
\textbf{Cargo:}
\\
\textbf{Assinatura:}


\end{document}